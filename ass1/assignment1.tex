\documentclass[10pt, a4paper]{article}

\usepackage{amssymb, amsmath}
\usepackage[utf8]{inputenc}
%\usepackage{ngerman}
%\usepackage[ngerman]{babel}
\usepackage[english]{babel}
\usepackage{fancyhdr}
\usepackage{tikz}
\usepackage{fullpage}
\usepackage{graphicx}
\usepackage{alltt}
\usepackage{rotating}
\usepackage{multirow}

\usepackage{caption}
\usepackage{subcaption}
\usepackage{float}
\usepackage{makecell}
\usepackage{hyperref}
\usepackage{indentfirst}

% Diagrams
\usepackage{tikz}
\usetikzlibrary{%
  arrows,%
  shapes.misc,% wg. rounded rectangle
  shapes.arrows,%
  shapes.geometric,% regular polygon
  chains,%
  matrix,%
  positioning,% wg. " of "
  scopes,%
  decorations.pathmorphing,% /pgf/decoration/random steps | erste Graphik
  shadows%
}

% Trace
\usepackage{array}   % for \newcolumntype macro
\newcounter{magicrownumbers} % counts trace statement number
\def\rownumber{\stepcounter{magicrownumbers}\arabic{magicrownumbers}} % print number
\newcolumntype{L}{>{(}l<{.}}
\newcolumntype{V}{>{\ttfamily}l<{)\textsuperscript{\rownumber},}}
\newcolumntype{P}{>{\it}l}
\newcolumntype{N}{r<{\rownumber}}
\newcolumntype{C}{>{\ttfamily\-}p{10cm}}
\newcolumntype{R}{r<{\textsuperscript{\rownumber}}}
\def\nodecounter#1{\tkmark{#1\twodigits{\arabic{magicrownumbers}}}}
%\newcommand*{\tknode}[1]{\tikz[overlay, remember picture]\node [inner sep=0, outer sep=0] (#1) {};}
\newcommand*{\tkmark}[1]{\tikz[overlay, remember picture]\coordinate (#1);}
\newcommand\twodigits[1]{%
   \ifnum#1<10 0#1\else #1\fi
}

\pagestyle{fancy}
\setlength{\headheight}{12.4pt}
\setlength{\headsep}{1.5\headheight}

% Übungsblatt-Nummer eintragen
\newcommand{\AssignmentNumber}{1}

% ݇Gruppen-Nummer eintragen
\newcommand{\GroupNumber}{12}

% 1. Person eintragen
\newcommand{\FirstAuthor}{Hamedl}
\newcommand{\FirstAuthorFirstName}{Tobias}
\newcommand{\FirstAuthorMatnum}{11808141}

% 2. Person eintragen
\newcommand{\SecAuthor}{Joao}
\newcommand{\SecAuthorFirstName}{Gerson Miguel}
\newcommand{\SecAuthorMatnum}{11804202}

% 3. Person eintragen
\newcommand{\ThirdAuthor}{Khalil}
\newcommand{\ThirdAuthorFirstName}{Mahmoud}
\newcommand{\ThirdAuthorMatnum}{01651551}

%4. Person eintragen
\newcommand{\FourthAuthor}{Piber}
\newcommand{\FourthAuthorFirstName}{Constantin}
\newcommand{\FourthAuthorMatnum}{11921514}

\newcommand{\AuthorFront}{{\normalsize
\begin{tabular}{|c|c|c|} \hline
\textbf{Nachname} & \textbf{Vorname}       & \textbf{Matrikelnummer} \\ \hline \hline
\FirstAuthor      & \FirstAuthorFirstName  & \FirstAuthorMatnum      \\ \hline
\SecAuthor        & \SecAuthorFirstName    & \SecAuthorMatnum        \\ \hline
\ThirdAuthor      & \ThirdAuthorFirstName  & \ThirdAuthorMatnum      \\ \hline
\FourthAuthor     & \FourthAuthorFirstName & \FourthAuthorMatnum     \\ \hline
\end{tabular}}}

\author{\AuthorFront}
\newcommand{\Author}{\FirstAuthorMatnum, \SecAuthorMatnum,
                     \ThirdAuthorMatnum, \FourthAuthorMatnum}

\date{} % Kein Datum angegeben
\fancyfoot{} % Seitenzahl unten nicht anzeigen

\lhead{Blatt \AssignmentNumber}
\chead{\Author}
\rhead{Seite \thepage}

\title{Software Maintenance SS 19/20, Assignment \AssignmentNumber}

\begin{document}

%%%%%%%%%%%%%%%%%%%%%%%%%%%%%%%%%%%%%%%%%%%%%%%%%%%%%%%%%%%%%%%%%%%%%%%%%%%%%%%%

\newcounter{ale}
\newcommand{\abc}{\item[\alph{ale})]\stepcounter{ale}}
\newenvironment{liste}{\begin{itemize}}{\end{itemize}}
\newcommand{\aliste}{\begin{liste} \setcounter{ale}{1}}
\newcommand{\zliste}{\end{liste}}
\newenvironment{abcliste}{\aliste}{\zliste}
%%%%%%%%%%%%%%%%%%%%%%%%%%%%%%%%%%%%%%%%%%%%%%%%%%%%%%%%%%%%%%%%%%%%%%%%%%%%%%%%

\maketitle
\begin{center}
 GROUP \textless \GroupNumber \textgreater
\end{center}
\thispagestyle{fancy}

%%%%%%%%%%%%%%%%%%%%%%%%%%%%%%%%%%%%%%%%%%%%%%%%%%%%%%%%%%%%%%%%%%%%%%%%%%%%%%%%
\begin{table}[h!]
\centering
\begin{tabular}{|l|l|l|c|c|c|c|}
\hline
\multicolumn{3}{|c|}{\textbf{Assignment}} & \textbf{max. points} & \textbf{received points} & \textbf{max. points} & \textbf{received points}  \\[2ex]
\hline
\multirow{10}{*}{Theory}&2.2.1&a & 2,5 &   & \multirow{3}{*}{} & \\[2ex] \cline{2-5}
&&b & 1,0 &  &  3,5  & \\[2ex] \cline{2-7}
&2.2.2& a & 0,5 & &  & \\[2ex] \cline{2-5}
&& b & 1,0 & &  1,5  & \\[2ex] \cline{2-7}
&2.2.3& a & 0,5 & &  & \\[2ex]\cline{2-5}
&& b & 0,75 & &  & \\[2ex]\cline{2-5}
&& c & 0,75 & & 2,0 & \\[2ex]\cline{2-7}
&2.2.4& a & 2,0 & &  & \\[2ex]\cline{2-5}
&& b & 1,0 & & 3,0 & \\[2ex]\cline{2-7}
\hline
\multirow{3}{*}{Programming}&Static slicer& & 9,0 &  & \multirow{3}{*}{10} & \\[2ex] \cline{2-5}
&Bug database& & 1,0 &  &  & \\[2ex] \cline{2-5}
\hline
\multicolumn{5}{|l|}{Total points}& 20 &\\[2ex]
\hline
\end{tabular}
\end{table}
%%%%%%%%%%%%%%%%%%%%%%%%%%%%%%%%%%%%%%%%%%%%%%%%%%%%%%%%%%%%%%%%%%%%%%%%%%%%%%%%
\newpage

\section{Theoretical Part}

\iffalse
\begin{enumerate}
 \item Format example:
%\vspace{10pt}
\begin{table}[htbp]
	\centering
		\begin{tabular}{|c|c|c|c|c|c|c|c|c|}
\hline
n & $PRE(n)$ & $REF(n)$ & $DEF(n)$ & $R_{(9,\{z\})}(n)$ & $S_{(9,\{z\})}^0(n)$ & $INFL(n)$   & $B$ & $S_{(9,\{z\})}^1(n)$\\ \hline
2 & - & -         & \{x\}     & \{y, z\}      &  x  &   &     & x \\ \hline
3 & 2 & \{y\}     & -         & \{x, y, z\}   &     & 5 & x   & x \\ \hline
& & & & & & & & \\ \hline
		\end{tabular}
\end{table}
\begin{table}[htbp]
    \centering
        \begin{tabular}{|c|c|c|c|c|c|c|c|c|c|c|c|c|c|c|}
\hline
n & \begin{turn}{90}$PRE(n)$ \end{turn} & \begin{turn}{90}$REF(n)$
\end{turn}& \begin{turn}{90}$DEF(n)$ \end{turn}&
\begin{turn}{90}$R_{(9,\{z\})}(n)$ \end{turn}&
\begin{turn}{90}$S_{(9,\{z\})}^0(n)$ \end{turn} &
\begin{turn}{90}$INFL(n)$ \end{turn}  & $B$ &
\begin{turn}{90}$S_{(9,\{z\})}^1(n)$ \end{turn} &
\begin{turn}{90}$R_{(9,\{z\})}(n)$\end{turn}&
\begin{turn}{90}$S_{(9,\{z\})}^0(n)$ \end{turn} &
\begin{turn}{90}$R_{(9,\{z\})}(n)$\end{turn}&
\begin{turn}{90}$S_{(9,\{z\})}^0(n)$ \end{turn} &
\begin{turn}{90}$R_{(9,\{z\})}(n)$\end{turn}&
\begin{turn}{90} $S_{(9,\{z\})}^0(n)$ \end{turn}\\ \hline
2 & - &\{result\} & \{x,y\}  & \{result,x,y,z\} &x  &  21-25  &  x &
\{x,y\} & x & \{x\}& x& \{x,y,z,w\} & x & x \\ \hline
3 & 2 & \{y\}     & -         & \{x, y, z\}   &     & 5 & x   & x & & &
& & &\\ \hline
& & & & & & & & & & & & & &\\ \hline
        \end{tabular}
\end{table} 
\end{enumerate}
\pagebreak
\fi


%%%%%%%% solution - theoretical part %%%%%%%%%



%%%% Tikz stuff
\tikzset{
  circ/.style={
    % The shape:
    circle,
    minimum size=6mm,
    % The rest
    very thick,draw=black,
    %fill=white,
    font=\ttfamily},
  ell/.style={
    % The shape:
    ellipse,
    %minimum width=6mm,
    minimum height=6mm,
    % The rest
    inner sep=0,
    draw=black,
    %fill=white,
    font=\ttfamily},
  stmt/.style={
    % The shape:
    rectangle,
    minimum size=6mm,
    % The rest
    very thick,draw=black,
    %fill=white,
    font=\ttfamily},
  cond/.style={
    % The shape:
    %regular polygon, regular polygon sides=4,
  	%shape border rotate=45,
  	%inner sep=0, outer sep=0,
  	rounded rectangle,
    minimum size=6mm,
    % The rest
    very thick,draw=black,
    %fill=white,
    font=\ttfamily},
  help/.style={
    minimum size=6mm},
  %
  ctrl/.style={->, draw=black, thick},
  data/.style={->, draw=red, dashed, thick},
  symm/.style={<->, draw=black!30!blue, dotted, thick},
  potd/.style={->, draw=black!40!green, dotted, thick},
  %
  markF/.style={
    execute at end node=\space {\Large $\bullet$}},
  markI/.style={
    execute at end node=\space {\color{red}\Large $\bullet$}},
  %
  setT/.style={
    draw=green},
  setX/.style={
    draw=red},
  %
  skip loop/.style={to path={-- ++(0,#1) -| (\tikztotarget)}},
  fuzzy/.style={decorate,
    decoration={random steps,segment length=0.5mm,amplitude=0.15pt}},
}

{
  \tikzset{circ/.append style={text height=1.5ex,text depth=.25ex}}
  \tikzset{ell/.append style={text height=1.5ex,text depth=.25ex}}
  \tikzset{stmt/.append style={text height=1.5ex,text depth=.25ex}}
  \tikzset{cond/.append style={text height=1.5ex,text depth=.25ex}}
}






\subsection{Static Slicing}

\begin{table}[H]
\centering
\caption*{\raggedright (a) i. Slice using table algorithm \quad {\ttfamily (20,\{a\})}}
\begin{tabular}{|c|*8{>{\ttfamily}c|}}
\hline
n & $PRE(n)$ & $REF(n)$ & $DEF(n)$ & $R_{(20,\{a\})}(n)$ & $S_{(20,\{a\})}^0(n)$ & $INFL(n)$   & $B$ & $S_{(20,\{a\})}^1(n)$\\ \hline
2  & -     & b   & a & b   & x &     &   & x \\ \hline
3  & 2     & a,d & c & a,b &   &     &   &   \\ \hline
4  & 3,7   & a,c & - & a,b &   & 6,7 & x & x \\ \hline
6  & 4     & a,b & a & a,b & x &     &   & x \\ \hline
7  & 6     & a,c & d & a,b &   &     &   &   \\ \hline
10 & 4     & a,c & - & a   &   & 12  &   &   \\ \hline
12 & 10    & a   & b & a   &   &     &   &   \\ \hline
14 & 10,12 & -   & - & a   &   & 16  &   &   \\ \hline
16 & 14    & c   & c & a   &   &     &   &   \\ \hline
19 & 14,16 & a,c & d & a   &   &     &   &   \\ \hline
20 & 19    & a   & - & a   &   &     &   & x \\ \hline
\end{tabular}
\caption*{\ttfamily s = \{2,4,6,20\}}
\end{table}

\begin{table}[H]
\centering
\caption*{\raggedright (a) ii. Slice using table algorithm \quad {\ttfamily (21,\{d\})}}
\begin{tabular}{|c|*8{>{\ttfamily}c|}}
\hline
n & $PRE(n)$ & $REF(n)$ & $DEF(n)$ & $R_{(21,\{d\})}(n)$ & $S_{(21,\{d\})}^0(n)$ & $INFL(n)$   & $B$ & $S_{(21,\{d\})}^1(n)$\\ \hline
2  & -     & b   & a & b,d & x &     &   & x \\ \hline
3  & 2     & a,d & c & a,d & x &     &   & x \\ \hline
4  & 3,7   & a,c & - & a,c &   & 6,7 &   & x \\ \hline
6  & 4     & a,b & a & a,c & x &     &   & x \\ \hline
7  & 6     & a,c & d & a,c &   &     &   &   \\ \hline
10 & 4     & a,c & - & a,c &   & 12  &   &   \\ \hline
12 & 10    & a   & b & a,c &   &     &   &   \\ \hline
14 & 10,12 & -   & - & a,c &   & 16  &   & x \\ \hline
16 & 14    & c   & c & a,c & x &     &   & x \\ \hline
19 & 14,16 & a,c & d & a,c & x &     &   & x \\ \hline
20 & 19    & a   & - & d   &   &     &   &   \\ \hline
21 & 20    & d   & - & d   &   &     &   & x \\ \hline
\end{tabular}
\caption*{\ttfamily s = \{2,3,4,6,14,16,19,21\}}
\end{table}

\begin{figure}[H]
\centering
\caption*{\raggedright (b) Slice using Program Dependence Graph}
\begin{tikzpicture}[
  dot/.style={circle, thin, -latex, yshift=4pt, scale=0.6},
   i/.style={dot, xshift=-3mm, fill=black!25!yellow!60!red},
  ii/.style={dot, xshift= 3mm, fill=black!20!green}
]
  % graph with control dependencies
  \begin{scope}[
    every node/.append style={
      grow=down
    },
    every path/.append style={ctrl}
  ]
    \node [ell] (entry) {Entry}
      [sibling distance=2cm]
      child { node [ell,yshift=9mm]  (stmt02) {2. a = b * 2} }
      child { node [ell,xshift=-9mm] (stmt03) {3. c = a + d} }
      child { node [ell,yshift=-9mm,xshift=-14mm] (stmt04) {4. while(a < c+5)}
        [sibling distance=36mm]
        child { node [ell] (stmt06) {6. a = a + b} }
        child { node [ell] (stmt07) {7. d = c - a} }
      }
      child { node [ell,yshift=-18mm,xshift=16mm] (stmt10) {10. if(a > c)}
        [sibling distance=4cm]
        child { node [ell] (stmt12) {12. b = a + 2} }
        child { node [ell] (stmt16) {16. c = c - 10} }
      }
      child { node [ell,yshift=-9mm,xshift=14mm] (stmt19) {19. d = a + c} }
      child { node [ell,xshift=9mm]  (stmt20) {20. write(a)} }
      child { node [ell,yshift=9mm]  (stmt21) {21. write(d)} };
  \end{scope}
  
  % data dependencies
  \begin{scope}
    % straight
    \draw [data] (stmt02) -- (stmt03);
    \draw [data] (stmt03) -- (stmt04);
    \draw [data] (stmt06) -- (stmt07);
    \draw [data] (stmt06.100) -- (stmt04.200);
    \draw [data] (stmt06.320) -- (stmt12.180);
    
    % bent
    \draw [data] (stmt02.0) to [bend left]  (stmt07.6);
    \draw [data] (stmt02.0) to [bend left]  (stmt10.160);
    \draw [data] (stmt02.0) to [bend left]  (stmt19.170);
    \draw [data] (stmt02.10) to [bend left] (stmt20.160);
    
    \draw [data] (stmt03.0) to [bend left]  (stmt07.6);
    \draw [data] (stmt03.0) to [bend left]  (stmt10.160);
    \draw [data] (stmt03.0) to [bend left]  (stmt19.170);
    
    \draw [data] (stmt06.10) to [bend left] (stmt10.175);
    
    \draw [data] (stmt19.160) to [bend left](stmt21.180);
    
    \draw [data] (stmt02.190) to[bend right](stmt04.180);
    \draw [data] (stmt02.190) to[bend right](stmt06.160);
    
    %\draw [data] (stmt02.190) to[bend right]([xshift=-2mm]stmt06.180) to [bend right] (stmt12.190);
    \draw [data] (stmt02.190) .. controls ([xshift=-7mm,yshift=-10mm]stmt06.180) .. (stmt12.180);
    
    %\draw [data] (stmt03.190) to[bend right]([xshift=-1mm]stmt06.180) to [bend right] (stmt16.190);
    \draw [data] (stmt03.185) .. controls ([xshift=-12mm,yshift=-15mm]stmt06.180) and
      ([xshift=-5mm,yshift=-25mm]stmt06.180) .. (stmt16.190);
      
    \draw [data] (stmt06.320) .. controls ([xshift=10mm,yshift=-18mm]stmt12.270) and
      ([xshift=41mm,yshift=-18mm]stmt16.270)  .. (stmt20.345);
  \end{scope}
  
  % slice dots
  \begin{scope}
    \node [i,  below=0mm of entry ] {};
    \node [ii, below=0mm of entry ] {};
    \node [i,  below=0mm of stmt02] {};
    \node [ii, below=0mm of stmt02] {};
    \node [ii, below=0mm of stmt03] {};
    \node [i,  below=0mm of stmt04] {};
    \node [ii, below=0mm of stmt04] {};
    \node [i,  below=0mm of stmt06] {};
    \node [ii, below=0mm of stmt06] {};
    \node [ii, below=0mm of stmt10] {};
    \node [ii, below=0mm of stmt16] {};
    \node [i,  below=0mm of stmt20] {};
    \node [ii, below=0mm of stmt21] {};
  \end{scope}
  
  % legend
  \begin{scope}
    \node [font=\ttfamily, below=65mm of entry]  (legendi)
      { i. (20, \{a\}) \quad s = \{2,4,6,20\}};
    \node [font=\ttfamily, below=0mm of legendi] (legendii)
      {ii. (21, \{d\}) \quad s = \{2,3,4,6,10,14,16,19,21\}};
    \node [i,  left=1mm of legendi, xshift= 3mm, yshift=-4pt]  {};
    \node [ii, left=1mm of legendii, xshift=-3mm, yshift=-4pt] {};
  \end{scope}

\end{tikzpicture}
\end{figure}

\pagebreak



\subsection{Control Flow Graph}

\begin{figure}[H]
\centering
\caption*{\raggedright (a)}
\begin{tikzpicture}[
  /pgf/every decoration/.style={/tikz/sharp corners},
]
  \draw[start chain=1 going below,
    node distance=4mm,
    every node/.append style={on chain=1},
  ]
      node [circ] (start)   {}
      node [stmt] (stmt1)   {a = b * 2}
      node [stmt] (stmt2)   {c = a + d}
      node [cond] (while)   {while (a < c + 5)}
      node [stmt] (stmt3)   {a = a + b}
      node [stmt] (stmt4)   {d = c - a}
      node [cond] (if)      {if (a > c)}
      node [help] (ifblock) {\-\ } % helper
      node [stmt] (stmt7)   {d = a + c}
      node [stmt] (stmt8)   {write(a)}
      node [stmt] (stmt9)   {write(d)}
      node [circ] (end)     {};
  \node [stmt, left=7mm of ifblock]  (stmt5) {b = a + 2};
  \node [stmt, right=7mm of ifblock] (stmt6) {c = c - 10};
  % helper
  \node [left=0mm of while]  (lwhile) {};
  \node [right=0mm of while] (rwhile) {};
  \node [above=2mm of stmt7] (ifafter) {};
  
  \draw [->] (start) -- (stmt1);
  \draw [->] (stmt1) -- (stmt2);
  \draw [->] (stmt2) -- (while);
  \draw [->] (while) -- node [right] {True} (stmt3);
  \draw [->] (stmt3) -- (stmt4);
  \draw [->] (stmt4.east) -| (rwhile.east) -- (while.east);
  \draw [->] (while.west) -- (lwhile.west) |-
    node[left, pos=0.25] {False} (if.175);
  \draw [->] (if.355) -| node [right] {False} (stmt6.north);
  \draw [->] (if.185) -| node [left] {True} (stmt5.north);
  \draw [  ] (stmt6.south) |- (ifafter.south);
  \draw [  ] (stmt5.south) |- (ifafter.south);
  \draw [->] (ifafter) -- (stmt7);
  \draw [->] (stmt7) -- (stmt8);
  \draw [->] (stmt8) -- (stmt9);
  \draw [->] (stmt9) -- (end);
  
\end{tikzpicture}
\end{figure}
\vspace{-\floatsep}

\begin{figure}[H]
\centering
\caption*{\raggedright (b)}
\begin{tikzpicture}[
  /pgf/every decoration/.style={/tikz/sharp corners},
]
  \draw[start chain=1 going below,
    node distance=4mm,
    every node/.append style={on chain=1},
  ]
      node [circ] (start)   {}
      node [stmt] (stmt1)   {a = m * o}
      node [stmt] (stmt2)   {s = m + a}
      node [cond] (if1)      {if (o < i)}
      node [help] (ifblock1){\-\ } % helper
      node [help] (ifblock2){\-\ } % helper
      node [help] (ifblock3){\-\ } % helper
      node [stmt] (stmt3)   {write(s)}
      node [stmt] (stmt4)   {write(m)}
      node [circ] (end)     {};
  % if true
  \draw[start chain=2 going below,
  	node distance=4mm,
  	every node/.append style={on chain=2}
  ]
      node [cond, left=25mm of ifblock1, anchor=center]  (if2) {if (x == 0)}
      node [help] (if2block1) {\-\ } % helper
      node [help] (if2block2) {\-\ }; % helper
  \node [stmt, left=12mm of if2block1, anchor=center]  (stmt1t) {x = x + 1};
  \node [stmt, left=12mm of if2block2, anchor=center]  (stmt2t) {s = o * 5};
  \node [stmt, right=12mm of if2block1, anchor=center] (stmt3t) {x = 10};
  % if false
  \draw[start chain=3 going below,
  	node distance=4mm,
  	every node/.append style={on chain=3}
  ]
  	  node [stmt, right=25mm of ifblock1, anchor=center] (stmt1f) {o = o + m}
  	  node [cond] (while)   {while (m <= a)}
  	  node [stmt] (stmt2f)  {m = x * s};
  % helper
  \node [left=0mm of while]  (lwhile) {};
  \node [right=0mm of while] (rwhile) {};
  \node [above=2mm of stmt3] (ifafter) {};
  
  \draw [->] (start) -- (stmt1);
  \draw [->] (stmt1) -- (stmt2);
  
  \draw [->] (stmt2) -- (if1);
  \draw [->] (if1.east) -| node[right] {False} (stmt1f.north);
  \draw [->] (if1.west) -| node[left] {True} (if2.north);
  
  \draw [->] (if2.east) -| node[right] {False} (stmt3t.north);
  \draw [->] (if2.west) -| node[left] {True} (stmt1t.north);
  \draw [->] (stmt1t) -- (stmt2t);
  \draw [  ] (stmt3t) |- node [pos=0.5] (iff) {} (ifafter.south);
  \draw [  ] (stmt2t) |- (iff.center);
  
  \draw [->] (stmt1f) -- (while);
  \draw [->] (while) -- node [right] {True} (stmt2f);
  \draw [->] (stmt2f.east) -| (rwhile.east) -- (while.east);
  \draw [  ] (while.west) -- (lwhile.west) |-
    node[left, pos=0.25] {False} (ifafter.south);
  
  \draw [->] (ifafter) -- (stmt3);
  \draw [->] (stmt3) -- (stmt4);
  \draw [->] (stmt4) -- (end);
  
\end{tikzpicture}
\end{figure}

\pagebreak



\subsection{Minimal Hitting Sets}

Discarded sets: red, \quad \quad Minimal hitting sets: green \\

\begin{figure}[H]
\centering
\caption*{\raggedright (a) {\ttfamily\{s,o\},\{m,a\},\{a,s,i,g\},\{o,n,e\}}}
\begin{tikzpicture}[
  every node/.append style={
    circ,
    grow=down
  },
  every path/.append style={thick,->}
]
  
  \node {}
    [sibling distance=8cm]
    child { node {s}
      [sibling distance=4cm]
      child { node {m}
        [sibling distance=1cm]
        child { node [setT] {o} }
        child { node [setT] {n} }
        child { node [setT] {e} }
      }
      child { node {a}
        [sibling distance=1cm]
        child { node [setX] {o} }
        child { node [setT] {n} }
        child { node [setT] {e} }
      }
    }
    child { node {o}
      [sibling distance=4cm]
      child { node {m}
        [sibling distance=1cm]
        child { node [setX] {a} }
        child { node [setX] {s} }
        child { node [setT] {i} }
        child { node [setT] {g} }
      }
      child { node [setT] {a} }
    };

\end{tikzpicture}
\caption*{\raggedright Minimal hitting sets: {\ttfamily\{o,a\},\{s,m,o\},\{s,m,n\},\{s,m,e\},\{s,a,n\},\{s,a,e\},\{o,m,i\},\{o,m,g\}}}
\end{figure}
\vspace{-\floatsep}

\begin{figure}[H]
\centering
\caption*{\raggedright (b) {\ttfamily\{t,u\},\{g,r,a,z\},\{a,s,u\},\{s,g,a\}}}
\begin{tikzpicture}[
  every node/.append style={
    circ,
    grow=down
  },
  every path/.append style={thick,->}
]

  \node {}
    [sibling distance=75mm]
    child { node {t}
      [sibling distance=21mm]
      child { node {g}
        [sibling distance=7mm]
        child { node [setX] {a} }
        child { node [setT] {s} }
        child { node [setX] {u} }
      }
      child { node {r}
        [sibling distance=7mm]
        child { node [setX] {a} }
        child { node [setT] {s} }
        child { node [setX] {u} }
      }
      child { node [setT] {a} }
      child { node {z}
        [sibling distance=7mm]
        child { node [setX] {a} }
        child { node [setT] {s} }
        child { node [setX] {u} }
      }
    }
    child { node {u}
      [sibling distance=21mm]
      child { node [setT] {g} }
      child { node [setT] {r} }
      child { node {a}
        [sibling distance=7mm]
        child { node [setT] {s} }
        child { node [setX] {g} }
        child { node [setX] {a} }
      }
      child { node {z}
        [sibling distance=7mm]
        child { node [setT] {s} }
        child { node [setX] {g} }
        child { node [setX] {a} }
      }
    };

\end{tikzpicture}
\caption*{\raggedright Minimal hitting sets: {\ttfamily\{t,a\},\{u,g\},\{u,r\},\{t,g,s\},\{t,r,s\},\{t,z,s\},\{u,a,s\},\{u,z,s\}}}
\end{figure}
\vspace{-\floatsep}

\begin{figure}[H]
\centering
\caption*{\raggedright (c) {\ttfamily\{c,a\},\{b,d,a\},\{c,d,e,f\},\{c,e,j\}}}
\begin{tikzpicture}[
  every node/.append style={
    circ,
    grow=down
  },
  every path/.append style={thick,->}
]
  
  \node {}
    [sibling distance=8cm]
    child { node {c}
      [sibling distance=26mm]
      child { node [setT] {b} }
      child { node [setT] {d} }
      child { node [setT] {a} }
    }
    child { node {a}
      [sibling distance=2cm]
      child { node [setX] {c} }
      child { node {d}
        [sibling distance=1cm]
        child { node [setX] {c} }
        child { node [setX] {e} }
        child { node [setT] {j} }
      }
      child { node [setT] {e} }
      child { node {f}
        [sibling distance=1cm]
        child { node [setX] {c} }
        child { node [setX] {e} }
        child { node [setT] {j} }
      }
    };

\end{tikzpicture}
\caption*{\raggedright Minimal hitting sets: {\ttfamily\{c,b\},\{c,d\},\{c,a\},\{a,e\},\{a,d,j\},\{a,f,j\}}}
\end{figure}
\pagebreak



\subsection{Dynamic Slicing}

Slicing criterion: \verb|({y=2,t=1}, 29|\textsuperscript{20}\verb|, {z})| \\

\begin{figure}[H]
\centering
\setcounter{magicrownumbers}{0}
\begin{tabular}{LVP}
2  & z = 20 \\                                % z = 20
3  & x = 5 \\                                 % x = 5
4  & read(y) \\                               % y = 2
5  & read(t) \\                               % t = 1
6  & while (y<5) \\                           % true
8  & y = y*2 \\                               % y = 4
9  & z = z - (y+x) \\                         % z = 11
6  & while (y<5) \\                           % true
8  & y = y*2 \\                               % y = 8
9  & z = z - (y+x) \\                         % z = -2
6  & while (y<5)   	 & PR(6,F) = \{y,z\} \\   % false
12 & if (x>t)        & PR(12,T) = \{x\} \\    % true
14 & x = t*x \\                               % x = 5
15 & while (z<0) \\                           % true
17 & t = t+20 \\                              % t = 21
18 & z = t+y \\                               % z = 29
15 & while (z<0)     & PR(15,F) = \{t,z\} \\  % false
21 & t = x*10 + z \\                          % t = 79
28 & write(x) \\
29 & write(t) \\
\end{tabular}
\caption*{Extended Execution Trace with Potentially Relevant Variables (PR)}
\end{figure}

\begin{figure}[H]
\centering
\caption*{Initial statement marked with red dot, rest of slice with black dot.}
\setcounter{magicrownumbers}{0}
\begin{tikzpicture}[
    >=latex,thick,
    /pgf/every decoration/.style={/tikz/sharp corners},
    line join=round,line cap=round,
  ]
    \begin{scope}[start chain=1 going below,
            node distance=1mm,
            every node/.append style={on chain, scale=0.9,
                minimum width=6cm, align=center},
            ell/.append style={
                execute at begin node=(,
                execute at end node=)\textsuperscript{\rownumber}},
        scale=0.9
        ]
        \node [ell, markF] (stmt01) {2.  z = 20};
        \node [ell, markF] (stmt02) {3.  x = 5};
        \node [ell, markF] (stmt03) {4.  read(y)};
        \node [ell, markF] (stmt04) {5.  read(t)};
        \node [ell, markF] (stmt05) {6.  while (y<5)};
        \node [ell, markF] (stmt06) {8.  y = y*2};
        \node [ell, markF] (stmt07) {9.  z = z - (y+x)};
        \node [ell, markF] (stmt08) {6.  while (y<5)};
        \node [ell, markF] (stmt09) {8.  y = y*2};
        \node [ell, markF] (stmt10) {9.  z = z - (y+x)};
        \node [ell, markF] (stmt11) {6.  while (y<5)};
        \node [ell, markF] (stmt12) {12. if (x>t)};
        \node [ell]        (stmt13) {14. x = t*x};
        \node [ell, markF] (stmt14) {15. while (z<0)};
        \node [ell, markF] (stmt15) {17. t = t+20};
        \node [ell, markI] (stmt16) {18. z = t+y};
        \node [ell, markF] (stmt17) {15. while (z<0)};
        \node [ell]        (stmt18) {21. t = x*10 + z};
        \node [ell]        (stmt19) {28. write(x)};
        \node [ell]        (stmt20) {29. write(t)};
        
        \node {};
        \node (legendctrl) {Control Dependency};
        \node (legenddata) {Data Dependency};
        \node (legendsymm) {Symmetric Dependency};
        \node (legendpotd) {Potential Data Dependency};
    \end{scope}
    \node [left=of legendctrl] (p1) {};
    \node [left=of legenddata] (p2) {};
    \node [left=of legendsymm] (p3) {};
    \node [left=of legendpotd] (p4) {};
    
    \begin{scope}[->,
            decoration={pre length=6pt, post length=6pt},
            rounded corners=2mm,
            every path/.style=fuzzy
        ]
        % Control dependencies
        % while 5
        \draw [ctrl]  (stmt05.east)  to [bend left=80]   (stmt06.east);
        \draw [ctrl]  (stmt05.east)  to [bend left=80]   (stmt07.east);
        % while 8
        \draw [ctrl]  (stmt08.east)  to [bend left=80]   (stmt09.east);
        \draw [ctrl]  (stmt08.east)  to [bend left=80]   (stmt10.east);
        % if 12
        \draw [ctrl]  (stmt12.east)  to [bend left=80]   (stmt13.east);
        \draw [ctrl]  (stmt12.east)  to [bend left=80]   (stmt13.east);
        \draw [ctrl]  (stmt12.east)  to [bend left=80]   (stmt14.east);
        \draw [ctrl]  (stmt12.east)  to [bend left=80]   (stmt17.east);
        \draw [ctrl]  (stmt12.east)  to [bend left=80]   (stmt18.east);
        % while 14
        \draw [ctrl]  (stmt14.east)  to [bend left=80]   (stmt15.east);
        \draw [ctrl]  (stmt14.east)  to [bend left=80]   (stmt16.east);
        
        % Data depencies
        % z 1
        \draw [data]  (stmt01.west)  to [bend right=80]  (stmt07.west);
        % x 2
        \draw [data]  (stmt02.west)  to [bend right=80]  (stmt07.west);
        \draw [data]  (stmt02.west)  to [bend right=80]  (stmt10.west);
        \draw [data]  (stmt02.west)  to [bend right=80]  (stmt12.west);
        \draw [data]  (stmt02.west)  to [bend right=80]  (stmt13.west);
        % y 3
        \draw [data]  (stmt03.west)  to [bend right=80]  (stmt05.west);
        \draw [data]  (stmt03.west)  to [bend right=80]  (stmt06.west);
        % t 4
        \draw [data]  (stmt04.west)  to [bend right=80]  (stmt12.west);
        \draw [data]  (stmt04.west)  to [bend right=80]  (stmt13.west);
        \draw [data]  (stmt04.west)  to [bend right=80]  (stmt15.west);
        % y 6
        \draw [data]  (stmt06.west)  to [bend right=80]  (stmt07.west);
        \draw [data]  (stmt06.west)  to [bend right=80]  (stmt08.west);
        \draw [data]  (stmt06.west)  to [bend right=80]  (stmt09.west);
        % z 7
        \draw [data]  (stmt07.west)  to [bend right=80]  (stmt10.west);
        % y 9
        \draw [data]  (stmt09.west)  to [bend right=80]  (stmt10.west);
        \draw [data]  (stmt09.west)  to [bend right=80]  (stmt11.west);
        \draw [data]  (stmt09.west)  to [bend right=80]  (stmt16.west);
        % z 10
        \draw [data]  (stmt10.west)  to [bend right=80]  (stmt14.west);
        % x 13
        \draw [data]  (stmt13.west)  to [bend right=80]  (stmt18.west);
        \draw [data]  (stmt13.west)  to [bend right=80]  (stmt19.west);
        % t 15
        \draw [data]  (stmt15.west)  to [bend right=80]  (stmt16.west);
        % z 16
        \draw [data]  (stmt16.west)  to [bend right=80]  (stmt17.west);
        \draw [data]  (stmt16.west)  to [bend right=80]  (stmt18.west);
        % t 18
        \draw [data]  (stmt18.west)  to [bend right=80]  (stmt20.west);
        
        % Symmetric Dependencies
        % 6
        \draw [symm]  (stmt05.east)  to [bend left=80]   (stmt08.east);
        \draw [symm]  (stmt08.east)  to [bend left=80]   (stmt11.east);
        % 15
        \draw [symm]  (stmt14.east)  to [bend left=80]   (stmt17.east);
        
        % Potential Data Dependencies
        % PR(6,FALSE)
        % z
        \draw [potd]  (stmt11.east)  to [bend left=80]   (stmt14.east);
        % y
        \draw [potd]  (stmt11.east)  to [bend left=80]   (stmt16.east);
        % PR(12,TRUE)
        % x
        \draw [potd]  (stmt12.east)  to [bend left=80]   (stmt13.east);
        % PR(15,FALSE)
        % z
        \draw [potd]  (stmt17.east)  to [bend left=80]   (stmt18.east);
        
        % Legend
        \draw [ctrl]  (p1) -- (legendctrl);
        \draw [data]  (p2) -- (legenddata);
        \draw [symm]  (p3) -- (legendsymm);
        \draw [potd]  (p4) -- (legendpotd);
    \end{scope}
\end{tikzpicture}
\vspace{5mm}
\caption*{Extended Execution Trace Graph with Slice marked}
\vspace{-2mm}
\caption*{Note: Dynamic (Terminal) and Relevant Slice are the same in this example}
\end{figure}

\begin{figure}[H]
\centering
\bgroup
\setlength{\tabcolsep}{1.1mm}
\setcounter{magicrownumbers}{0}
\begin{tabular}{%
  |>{\footnotesize}R >{\ttfamily\footnotesize}l%
  |>{\footnotesize}c%
  |>{\nodecounter{data}}m{26mm}%
  |>{\nodecounter{ctrl}}l%
  |>{\nodecounter{symm}}l%
  |>{\nodecounter{potd}}l%
  |*2{c|}%
}
\hline
\gdef\rownumber{}%
& \footnotesize\makecell{Extended\\Execution\\Trace} %
& \footnotesize\makecell{Potential\\Relations} %
& \footnotesize\makecell{Data\\Dependencies} %
& \footnotesize\makecell{Control\\Dependencies} %
& \footnotesize\makecell{Symmetric\\Dependencies} %
& \footnotesize\makecell{Potential\\Dependencies} %
& \footnotesize\makecell{Terminal\\Slice} %
& \footnotesize\makecell{Relevant\\Slice} %
\gdef\rownumber{\stepcounter{magicrownumbers}\arabic{magicrownumbers}} \\
\hline
2  & z = 20 & & & & & & x & x \\\hline
3  & x = 5 & & & & & & x & x \\\hline
4  & read(y) & & & & & & x & x \\\hline
5  & read(t) & & & & & & x & x \\\hline
6  & while(y<5) & & & & & & x & x \\\hline
8  & y = y*2 & & & & & & x & x \\\hline
9  & z = z-(y+x) & & & & & & x & x \\\hline
6  & while(y<5) & & & & & & x & x \\\hline
8  & y = y*2 & & & & & & x & x \\\hline
9  & z = z-(y+x) & & & & & & x & x \\\hline
6  & while(y<5)     & \makecell{PR(6,F)\\= \{y,z\}} & & & & & x & x \\\hline
12 & if (x>t)       & \makecell{PR(12,T)\\= \{x\}} & & & & & x & x \\\hline
14 & x = t*x & & & & & & & \\\hline
15 & while(z<0) & & & & & & x & x \\\hline
17 & t = t+20 & & & & & & x & x \\\hline
18 & z = t+y & & & & & & x & x \\\hline
15 & while(z<0)     & \makecell{PR(15,F)\\= \{t,z\}} & & & & & x & x \\\hline
21 & t = x*10+z & & & & & & & \\\hline
28 & write(x) & & & & & & & \\\hline
29 & write(t) & & & & & & & \\\hline
\end{tabular}
\caption*{Dynamic (Terminal) and Relevant Slice}
\egroup

\begin{tikzpicture}[overlay, remember picture,
  ->,
  decoration={pre length=6pt, post length=6pt},
  rounded corners=2mm,
  every path/.style=fuzzy,
  >=latex,thick,
  /pgf/every decoration/.style={/tikz/sharp corners},
  line join=round,line cap=round,
]
  % Data depencies
  % z 1
  \draw [data]  ([xshift=02mm]data01) -- ([xshift=02mm]data07);
  % x 2
  \draw [data]  ([xshift=18mm]data02) -- ([xshift=18mm]data07);
  \draw [data]  ([xshift=20mm]data02) -- ([xshift=20mm]data10);
  \draw [data]  ([xshift=22mm]data02) -- ([xshift=22mm]data12);
  \draw [data]  ([xshift=24mm]data02) -- ([xshift=24mm]data13);
  % y 3
  \draw [data]  ([xshift=04mm]data03) -- ([xshift=04mm]data05);
  \draw [data]  ([xshift=06mm]data03) -- ([xshift=06mm]data06);
  % t 4
  \draw [data]  ([xshift=08mm]data04) -- ([xshift=08mm]data12);
  \draw [data]  ([xshift=10mm]data04) -- ([xshift=10mm]data13);
  \draw [data]  ([xshift=12mm]data04) -- ([xshift=12mm]data15);
  % y 6
  \draw [data]  ([xshift=04mm]data06) -- ([xshift=04mm]data07);
  \draw [data]  ([xshift=14mm]data06) -- ([xshift=14mm]data08);
  \draw [data]  ([xshift=16mm]data06) -- ([xshift=16mm]data09);
  % z 7
  \draw [data]  ([xshift=06mm]data07) -- ([xshift=06mm]data10);
  % y 9
  \draw [data]  ([xshift=02mm]data09) -- ([xshift=02mm]data10);
  \draw [data]  ([xshift=04mm]data09) -- ([xshift=04mm]data11);
  \draw [data]  ([xshift=14mm]data09) -- ([xshift=14mm]data16);
  % z 10
  \draw [data]  ([xshift=16mm]data10) -- ([xshift=16mm]data14);
  % x 13
  \draw [data]  ([xshift=02mm]data13) -- ([xshift=02mm]data18);
  \draw [data]  ([xshift=04mm]data13) -- ([xshift=04mm]data19);
  % t 15
  \draw [data]  ([xshift=06mm]data15) -- ([xshift=06mm]data16);
  % z 16
  \draw [data]  ([xshift=08mm]data16) -- ([xshift=08mm]data17);
  \draw [data]  ([xshift=10mm]data16) -- ([xshift=10mm]data18);
  % t 18
  \draw [data]  ([xshift=06mm]data18) -- ([xshift=06mm]data20);
  
  % Control dependencies
  % while 5
  \draw [ctrl]  ([xshift=2mm]ctrl05) -- ([xshift=2mm]ctrl06);
  \draw [ctrl]  ([xshift=4mm]ctrl05) -- ([xshift=4mm]ctrl07);
  % while 8
  \draw [ctrl]  ([xshift=2mm]ctrl08) -- ([xshift=2mm]ctrl09);
  \draw [ctrl]  ([xshift=4mm]ctrl08) -- ([xshift=4mm]ctrl10);
  % if 12
  \draw [ctrl]  ([xshift=2mm]ctrl12) -- ([xshift=2mm]ctrl13);
  \draw [ctrl]  ([xshift=4mm]ctrl12) -- ([xshift=4mm]ctrl13);
  \draw [ctrl]  ([xshift=6mm]ctrl12) -- ([xshift=6mm]ctrl14);
  \draw [ctrl]  ([xshift=8mm]ctrl12) -- ([xshift=8mm]ctrl17);
  \draw [ctrl]  ([xshift=1cm]ctrl12) -- ([xshift=1cm]ctrl18);
  % while 14
  \draw [ctrl]  ([xshift=2mm]ctrl14) -- ([xshift=2mm]ctrl15);
  \draw [ctrl]  ([xshift=4mm]ctrl14) -- ([xshift=4mm]ctrl16);
  
  % Symmetric Dependencies
  % 6
  \draw [symm]  ([xshift=2mm]symm05) -- ([xshift=2mm]symm08);
  \draw [symm]  ([xshift=4mm]symm08) -- ([xshift=4mm]symm11);
  % 15
  \draw [symm]  ([xshift=2mm]symm14) -- ([xshift=2mm]symm17);
  
  % Potential Data Dependencies
  % PR(6,FALSE)
  % z
  \draw [potd]  ([xshift=2mm]potd11) -- ([xshift=2mm]potd14);
  % y
  \draw [potd]  ([xshift=4mm]potd11) -- ([xshift=4mm]potd16);
  % PR(12,TRUE)
  % x
  \draw [potd]  ([xshift=6mm]potd12) -- ([xshift=6mm]potd13);
  % PR(15,FALSE)
  % z
  \draw [potd]  ([xshift=2mm]potd17) -- ([xshift=2mm]potd18);
\end{tikzpicture}
\end{figure}

\begin{figure}[H]
\centering
\setcounter{magicrownumbers}{0}
\begin{tabular}{NC}
& {\bf begin} \\
& \quad z = 20; \\
& \quad x = 5; \\
& \quad read( y ); \\
& \quad read( t ); \\
& \quad {\bf while}( y < 5 ) {\bf do} \\
& \quad \quad {\bf begin} \\
& \quad \quad \quad y = y * 2; \\
& \quad \quad \quad z = z - (y + x); \\
& \quad \quad {\bf end;} \\
& \quad {\bf od;} \\
& \quad {\bf if}( x < t ) {\bf then} \\
& \quad \quad {\bf begin} \\
& \quad \quad \quad {\bf while}( z < 0 ) {\bf do} \\
& \quad \quad \quad \quad {\bf begin} \\
& \quad \quad \quad \quad \quad t = t + 20; \\
& \quad \quad \quad \quad \quad z = t + y; \\
& \quad \quad \quad \quad {\bf end;} \\
& \quad \quad \quad {\bf od;} \\
& \quad \quad {\bf end;} \\
& \quad {\bf fi;} \\
& {\bf end}
\end{tabular}
\caption*{Example reduced to Slice}
\end{figure}



\section{Practical Part}

Repo: \url{https://git-students.ist.tugraz.at/soma20/group-12}

Slicer Project (for IntelliJ IDEA): \verb|submission1/Slicer|

Executable File: \verb|submission1/Slicer/staticslicer_12.jar|

Built using \verb|openjdk version "1.8.0_242"|


\end{document}